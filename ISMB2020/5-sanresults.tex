\section{Results}

As a part of this paper, we conducted a series of results to compare the performance of the proposed method to the existing state-of-the-art methods.
The methods compared include LASSO and SVM-RFE.
For each of the experiments, features were selected in the range of 10 to 500.
In all the three methods, these selected features were used to train a linear classifier 




A series of experiements is conducted to compare the performance of CoRAE with other state-of-the-art feature selection methods such as LASSO and SVM-RFE. A range of features from 10 to 500 has been selected using all three methods, then train a linear classifier (SVM) using selected coding and non-coding gene expression of 33 cancer patients. Figure 3 shows the classification performance for different number of features. Across the all k, CoRAE has highest accuracy and lowest error for both mRNA and lncRNA expression. Even if the number of feature is low e.g. 10, the accuracy is almost 80% whereas LASSO and SVM- RFE shows poor restuls for lowest number of feature. For more than 50 features, CoRAE shows more than 90% accuracy. Also, it shows less error with less number of features compare to other methods. For mRNA, CoRAE starts with MSE of 38 and quickly recduced to less than 10 within top 100 features. The behaviour in classification is almost similar in both coding and non-coding genes. However, mRNA expression performs slightly better than lncRNA which as shown in Figure 4.

3.1 Interpreting Related Features
CoRAE not only able to identify important features but also allows the user to examine relevance by observing the estimated concrete parameter α(i) for each feature. Since CoRAE selects a feature based on the value of vector α(i), the user can check the importance of each feature and find the correlation with others. In Figure ??, it is visually revealed that the top 100 mRNA or lncRNA is capable of distinguishing 33 cancer types. Also, it is noticeable that the feature selected by CoRAE is carrying more information than among all other features. Thus, influential features are selected in proposed method.
